% ====== TAREA 3 MATEMATICAS APLICADAS ======
\documentclass{article}
\usepackage[utf8]{inputenc}
\usepackage[spanish]{babel}
\usepackage{amsmath, amsfonts, amssymb}
\usepackage{ tipa }
\usepackage{graphicx}
\usepackage[usenames]{color}
\usepackage[text={20cm,25cm},centering,top=1.5cm,bottom=1.5cm,letterpaper,showframe=false]{geometry}
\renewcommand{\baselinestretch}{1.5}
\parindent  = 0mm
\parskip    = 4mm
\definecolor{azul}{RGB}{10,80,190}
\definecolor{negro}{RGB}{0,0,0}
\definecolor{rojo}{RGB}{190,80,10}
\definecolor{verde}{RGB}{0,120,50}

\begin{document}
    \title{Tarea-Examen 3}
    \author{Careaga Carrillo Juan Manuel\\
            Quiróz Castañeda Edgar\\
            Soto Corderi Sandra del Mar}
    \date{Viernes 7 de diciembre de 2018}
    \maketitle
    \begin{enumerate}

        % Ejercicio 1
        \item {
            Demostrar que
            
            \begin{enumerate}
            \item{
				$\nabla \times (f \mathbf{F}) = f \nabla \times \mathbf{F} + \nabla f \times \mathbf{F}$
				
			\color{azul}
            % Respuesta
           

            }
            
            \item{
            	$\nabla \times (\nabla \times \mathbf{F}) = \nabla (\nabla \cdot \mathbf{F}) - \nabla^2 \mathbf{F}$
           
           \color{azul}
            % Respuesta
            
            }
            \end{enumerate}                        
	    }

        % Ejercicio 2
        \item {
            Determinar el rotacional y la divergencia de los campos
            
            \begin{enumerate}
            \item{
				$\mathbf{F} (x,y,z) = (0,\cos xz,\sin xy)$
				
			\color{azul}
            % Respuesta           
            }
            
            \item{
            	$\mathbf{F} (x,y,z) = (\frac{x}{x^2 + y^2 + z^2}, \frac{y}{x^2 + y^2 + z^2}, \frac{z}{x^2 + y^2 + z^2})$
           
           \color{azul}
            % Respuesta
            
            }
            \end{enumerate}            
        }

        % Ejercicio 3
        \item {
            Determinar si $\mathbf{F}$ es campo vectorial conservativo y en su caso encuentre el campo escalar $f$
            
            \begin{enumerate}
            \item{
				$\mathbf{F} (x,y) = (2x\cos y - y\cos x, -x^2\sin y -\sin x)$
				
			\color{azul}
            % Respuesta
           

            }
            
            \item{
            	$\mathbf{F} (x,y,z) = (2xy, x^2 + 2yz, y^2)$
           
           \color{azul}
            % Respuesta
            
            }
            \end{enumerate}           
        }

        % Ejercicio 4
        \item {
           Calcular $\int_{C} \mathbf{F} \cdot d \mathbf{l}$
            
            \begin{enumerate}
            \item{
				$\mathbf{F} (x,y,z) = (\sin x,\cos y,xz)$ y la parametrización $\sigma (t) = (t^3,-t^2,t)$
				
			\color{azul}
            % Respuesta
           

            }
            
            \item{
            	$\mathbf{F} (x,y) = (x-y,xy)$ y $C$ el arco de círculo $x^2 + y^2 = 4$ que se recorre en sentido antihorario de $(2,0)$ a $(0,-2)$
           
           \color{azul}
            % Respuesta
            
            }
            \end{enumerate}           
        }

        % Ejercicio 5
        \item {
            Calcular $\iint_{S} f dS$
            
            \begin{enumerate}
            \item{
				$f(x,y,z) = x^2y + z^2$ y $S$ la parte del cilindro $x^2 + y^2 = 9$ que está entre los planos $z=0$ y $z=2$
				
			\color{azul}
            % Respuesta
           

            }
            
            \item{
            	$f(x,y,z) = x^2yz$ y $S$ la parte del plano $z = 1 + 2x + 3y$ que está arriba del rectángulo $[0,3] \times [0,2]$
           
           \color{azul}
            % Respuesta
            
            }
            \end{enumerate}

            
	    }

        % Ejercicio 6
        \item {
           Calcular $\iint_{S} \mathbf{F} \cdot d \mathbf{S}$
            
            \begin{enumerate}
            \item{
				$\mathbf{F} (x,y,z) = (0,y,-z)$ y $S$ consiste en el paraoloide $y = x^2 + z^2$, $0 \leq y \leq 1$, y el disco $x^2 + z^2 \leq 1$, $y = 1$
				
			\color{azul}
            % Respuesta
           

            }
            
            \item{
            	$\mathbf{F} (x,y) = (xze^y,-xze^y,z)$ y $S$ la parte del plano $x + y+ z = 1$ en el primer octante con orientación hacia arriba
           
           \color{azul}
            % Respuesta
            
            }
            \end{enumerate} 

            
        }

        % Ejercicio 7
        \item {
            Usar el teorema de Stokes para evaluar $\iint_{S} \nabla \times \mathbf{F} \cdot d\mathbf{S}$, donde $\mathbf{F}(x,y,z) =(x^2y^3z,\sin(xyz),xyz)$ y $S$ es la parte del cono $y^2 = x^2 + z^2$ que está entre los planos $y = 0$ y $y = 3$ orientada en dirección positiva del eje Y.

            \color{azul}
            % Respuesta

        }

        % Ejercicio 8
        \item {
            Usar el teorema de Gauss para evaluar $\iint_{S} \mathbf{F} \cdot d\mathbf{S}$, con $\mathbf{F}(x,y,z) = (x^3y,-x^2y^2,-x^2yz)$ y $S$ la superficie cerrada definida por el hiperboloide $x^2 + y^2 - z^2 = 1$ y los planos $z = -2$ y $z = 2$

            \color{azul}
            % Respuesta
           
        }
        
         % Ejercicio 9
        \item {
            Usar propiedades del rotacional para mostrar que
            \[
                \int_C {{\left(f \nabla g + g \nabla f\right)}\,d\mathbf{l}} = 0
            \]
         

            \color{azul}
            % Respuesta
           
        }
    \end{enumerate}
\end{document}
